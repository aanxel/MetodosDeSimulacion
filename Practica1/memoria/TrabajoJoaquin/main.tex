%%%%%%%%%%%%%%%%%%%%%%%%%%%%%%%%%%%%%%%%%
% Lachaise Assignment
% LaTeX Template
% Version 1.0 (26/6/2018)
%
% This template originates from:
% http://www.LaTeXTemplates.com
%
% Authors: 
% Marion Lachaise & François Févotte
% Vel (vel@LaTeXTemplates.com)
%
% License:
% CC BY-NC-SA 3.0 (http://creativecommons.org/licenses/by-nc-sa/3.0/)
% 
%%%%%%%%%%%%%%%%%%%%%%%%%%%%%%%%%%%%%%%%%

%----------------------------------------------------------------------------------------
%	PACKAGES AND OTHER DOCUMENT CONFIGURATIONS
%----------------------------------------------------------------------------------------

\documentclass{article}
\usepackage[spanish]{babel}
\usepackage[utf8]{inputenc}
\usepackage{titling}
\usepackage{mathtools}
\DeclarePairedDelimiter\ceil{\lceil}{\rceil}
\DeclarePairedDelimiter\floor{\lfloor}{\rfloor}

%----------------------------------------------------------------------------------------
%	ASSIGNMENT INFORMATION
%----------------------------------------------------------------------------------------

\newcommand\course{\emph{Máster Universitario en Inteligencia Artificial \\ Seminario de Metodología de la Investigación}}

\title{Resumen de S1: Metodología de la Investigación} % Title of the assignment

\author{Joaquín Jiménez López de Castro --- \texttt{jo.jimenez@alumnos.upm.es}} % Author name and email address

%----------------------------------------------------------------------------------------

\input{structure.tex} % Include the file specifying the document structure and custom commands


\begin{document}

\onlytitle

%----------------------------------------------------------------------------------------
%	INTRODUCTION
%----------------------------------------------------------------------------------------

\section{Contraste Máximo-de-t}

El contraste está descrito en Knuth, (1988). Dada una muestra de números \(U=\{U_1,U_2,...U_n\}\), teniendo \(U_i \in (0, 1)\) para cualquier \(i\), sirve para rechazar la hipótesis de que \(U\) sigue una distribución uniforme en \((0,1)\). El procedimiento consiste dividir \(U\) en \(m=\floor*{\frac{n}{t}}\) \emph{clusters} \(C=\{C_1,...,C_m\}\), con \(C_j=\{U_{jt},U_{jt+1},...,U_{jt+t-1}\}\) con \(t\ge1, j=1,...,m\). Se obtiene \(V=\{max(C_1),...,max(C_m)\}=\{V_1,...V_m\}\). La hipótesis se rechaza si el test de Kolmogorov-Smirnov rechaza la hipótesis de que \(V\) tiene \(F(x)=x^t, 0\le x\le 1\) como función de distribución.

El motivo es que si \(U\) sigue una distribución uniforme, entonces:

\begin{align*}
    F(x)=P(V_j \le x)&=\\
                &=P(max(\{U_{jt},U_{jt+1},...,U_{jt+t-1}\})\le x)=\\
                &=P(U_{jt}\le x)P(U_{jt+1}\le x)...P(U_{jt+t-1}\le x)=xx...x=x^t
\end{align*}

Este contraste admite variaciones. Una implícita es el parámetro \(t\), donde si \(t=1\), se tiene una mera comprobación de que los valores de \(U\) son uniformes. En la librería TestU01 se usa \(t=6\). También se puede intercambiar el test de Kolmogorov-Smirnov por el contraste \(\chi^2\), que es lo que hace TestU01.

Es sencillo plantear una muestra que pase el test y tenga un patrón fácilmente observable, pero la utilidad de este contraste reside en rechazar generadores de números aleatorios, no en aceptarlos.  


\end{document}
