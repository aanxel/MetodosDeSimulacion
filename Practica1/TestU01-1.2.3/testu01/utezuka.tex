\defmodule {utezuka}

This module collects some generators designed by S.~Tezuka.

%%%%%%%%%%%%%%%%%%%%%%%%%%%%%%%%%%%%%%%%%%%%%%%%%%%%%%%%%%%%%%
\bigskip
\hrule
\code
\hide
/* utezuka.h for ANSI C */

#ifndef UTEZUKA_H
#define UTEZUKA_H
\endhide
#include "unif01.h"


unif01_Gen * utezuka_CreateTezLec91 (unsigned int Y1, unsigned int Y2);
\endcode
  \tab  Implements a combined Tausworthe generator constructed
   by Tezuka and L'Ecuyer \cite{rTEZ91b}, and whose implementation
   is given in their paper.
\index{Generator!Tezuka}%
   The initial values {\tt Y1} and {\tt Y2} must be positive and
   less than $2^{31}$ and $2^{29}$ respectively.
  \endtab
\code


unif01_Gen * utezuka_CreateTez95 (unsigned int Y1, unsigned int Y2,
                                  unsigned int Y3);
\endcode
  \tab  Implements the  combined  generator proposed in
   Figure A.1 of \cite{rTEZ95a}, page 194.
   The initial values {\tt Y1, Y2, Y3} must be positive and
   less than $2^{28}$, $2^{29}$ and $2^{31}$
   respectively.
  \endtab
\code


unif01_Gen * utezuka_CreateTezMRG95 (unsigned int Y1[5],
                                     unsigned int Y2[7]);
\endcode
  \tab  Implements the  combined  generator proposed in
   Figure A.2 of \cite{rTEZ95a}, page 195.
   The initial values of the array elements of {\tt Y1} and {\tt Y2}
   must be positive and
   less than $2^{31}$ and $2^{29}$ respectively.
  \endtab



\guisec{Clean-up functions}

\code

void utezuka_DeleteGen (unif01_Gen *gen);
\endcode
 \tab \DelGen
 \endtab
\code

\hide
#endif
\endhide
\endcode
